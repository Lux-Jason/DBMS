% Options for packages loaded elsewhere
\PassOptionsToPackage{unicode}{hyperref}
\PassOptionsToPackage{hyphens}{url}
%
\documentclass[
]{article}
\usepackage{amsmath,amssymb}
\usepackage{iftex}
\ifPDFTeX
  \usepackage[T1]{fontenc}
  \usepackage[utf8]{inputenc}
  \usepackage{textcomp} % provide euro and other symbols
\else % if luatex or xetex
  \usepackage{unicode-math} % this also loads fontspec
  \defaultfontfeatures{Scale=MatchLowercase}
  \defaultfontfeatures[\rmfamily]{Ligatures=TeX,Scale=1}
\fi
\usepackage{lmodern}
\ifPDFTeX\else
  % xetex/luatex font selection
\fi
% Use upquote if available, for straight quotes in verbatim environments
\IfFileExists{upquote.sty}{\usepackage{upquote}}{}
\IfFileExists{microtype.sty}{% use microtype if available
  \usepackage[]{microtype}
  \UseMicrotypeSet[protrusion]{basicmath} % disable protrusion for tt fonts
}{}
\makeatletter
\@ifundefined{KOMAClassName}{% if non-KOMA class
  \IfFileExists{parskip.sty}{%
    \usepackage{parskip}
  }{% else
    \setlength{\parindent}{0pt}
    \setlength{\parskip}{6pt plus 2pt minus 1pt}}
}{% if KOMA class
  \KOMAoptions{parskip=half}}
\makeatother
\usepackage{xcolor}
\usepackage{longtable,booktabs,array}
\usepackage{calc} % for calculating minipage widths
% Correct order of tables after \paragraph or \subparagraph
\usepackage{etoolbox}
\makeatletter
\patchcmd\longtable{\par}{\if@noskipsec\mbox{}\fi\par}{}{}
\makeatother
% Allow footnotes in longtable head/foot
\IfFileExists{footnotehyper.sty}{\usepackage{footnotehyper}}{\usepackage{footnote}}
\makesavenoteenv{longtable}
\setlength{\emergencystretch}{3em} % prevent overfull lines
\providecommand{\tightlist}{%
  \setlength{\itemsep}{0pt}\setlength{\parskip}{0pt}}
\setcounter{secnumdepth}{-\maxdimen} % remove section numbering
\ifLuaTeX
  \usepackage{selnolig}  % disable illegal ligatures
\fi
\usepackage{bookmark}
\IfFileExists{xurl.sty}{\usepackage{xurl}}{} % add URL line breaks if available
\urlstyle{same}
\hypersetup{
  hidelinks,
  pdfcreator={LaTeX via pandoc}}

\author{}
\date{}

\begin{document}

COMP3013 2024 Fall

Assignment 1

Note: there is no requirement on ER diagram drawing. You are allowed to
draw ER diagrams using any tool. But you need to make sure that your
drawing is clear enough. TAs have rights to remove marks if your graph
unreadable. Please do not try to challenge us.

For the submission, please pack your answers to Q2 and Q3 to \textbf{A}
\textbf{SINGLE PDF FILE}. Rename the PDF file as
COMP3013\_23F\_A1\_XXXX, where XXXX is your student ID. Answers of Q1
are submitted via \textbf{iSpace Feedback}. Submissions which do not
following the guideline may \textbf{not be marked}.

Q1. The schema of a database is given as follows. Keys are underlined.

\begin{itemize}
\item
  \(book = (\underline{bID},bname,year,author)\)
\item
  \(reviewer = (\underline{rID},rname)\)
\item
  \(rating = (\underline{rID,bID},score,date)\) //\(score\) is an
  integer and \(1 \leq score \leq 5\)
\item
  \(comment = (\underline{rID,bID},content)\)
\end{itemize}

Write a query for each following question. Please submit your answers of
this question via \textbf{iSpace Feedback}. (7 marks for each)

\begin{enumerate}
\def\labelenumi{\alph{enumi})}
\item
  Find the name of books which are written by Kleene (author).
\item
  Find the name of books which have received a rating score less than 3.
\item
  Find the name of reviewers who have made a comment to book ``Quantum
  Finance'' (book name).
\item
  Find the name of reviewers who have written some book.
\item
  Find the authors who have written multiple books.
\item
  Find the name of books which have been rated at both score 1 and score
  5.
\item
  Can one reviewer rate a book multiple times? Why?
\end{enumerate}

Q2. Suppose you are a medical school student. You want to design an ER
diagram to model illnesses, treatments, and medicines under the
following assumptions. (35 marks)

\begin{itemize}
\item
  Each illness is described as ID, name, and discoverer's name.
\item
  Every illness is associated with one or multiple cause(s).
\item
  Each cause is described as ID, name, and description.
\item
  Some causes are associated to one or multiple organ(s) to describe
  where the causes happen.
\item
  Each organ is defined by ID, name, and body system.
\item
  Every illness is associated with one or multiple symptom(s). Each
  association also describes how long the symptom occurs.
\item
  A symptom is defined by ID, name, and effect.
\item
  Each illness is treated by at most one treatment.
\item
  Each treatment is defined by ID, name, equipment, and method.
\item
  Each illness is given zero, one, or multiple medicine(s).
\item
  If a medicine is given to an illness, the quantity is modeled.
\item
  Each medicine is defined by ID, name, and ingredient.
\end{itemize}

You do not need to make more assumptions on your design.

Q3. Given the tables, what is the execution outcome (including both
table title and table content) of the following queries? (4 marks for
each)

\begin{longtable}[]{@{}
  >{\raggedright\arraybackslash}p{(\columnwidth - 4\tabcolsep) * \real{0.3333}}
  >{\raggedright\arraybackslash}p{(\columnwidth - 4\tabcolsep) * \real{0.3333}}
  >{\raggedright\arraybackslash}p{(\columnwidth - 4\tabcolsep) * \real{0.3333}}@{}}
\toprule\noalign{}
\begin{minipage}[b]{\linewidth}\raggedright
\[sID\]
\end{minipage} & \begin{minipage}[b]{\linewidth}\raggedright
\[sname\]
\end{minipage} & \begin{minipage}[b]{\linewidth}\raggedright
\[age\]
\end{minipage} \\
\midrule\noalign{}
\endhead
\bottomrule\noalign{}
\endlastfoot
101 & Alice & 19 \\
102 & Bob & 18 \\
\end{longtable}

Table 1 Student

\begin{longtable}[]{@{}
  >{\raggedright\arraybackslash}p{(\columnwidth - 4\tabcolsep) * \real{0.3333}}
  >{\raggedright\arraybackslash}p{(\columnwidth - 4\tabcolsep) * \real{0.3333}}
  >{\raggedright\arraybackslash}p{(\columnwidth - 4\tabcolsep) * \real{0.3333}}@{}}
\toprule\noalign{}
\begin{minipage}[b]{\linewidth}\raggedright
\[sID\]
\end{minipage} & \begin{minipage}[b]{\linewidth}\raggedright
\[course\]
\end{minipage} & \begin{minipage}[b]{\linewidth}\raggedright
\[score\]
\end{minipage} \\
\midrule\noalign{}
\endhead
\bottomrule\noalign{}
\endlastfoot
101 & Database & 80 \\
102 & Database & 65 \\
\end{longtable}

Table 2 Enroll

\begin{longtable}[]{@{}
  >{\raggedright\arraybackslash}p{(\columnwidth - 4\tabcolsep) * \real{0.3333}}
  >{\raggedright\arraybackslash}p{(\columnwidth - 4\tabcolsep) * \real{0.3333}}
  >{\raggedright\arraybackslash}p{(\columnwidth - 4\tabcolsep) * \real{0.3333}}@{}}
\toprule\noalign{}
\begin{minipage}[b]{\linewidth}\raggedright
\[sID\]
\end{minipage} & \begin{minipage}[b]{\linewidth}\raggedright
\[course\]
\end{minipage} & \begin{minipage}[b]{\linewidth}\raggedright
\[remark\]
\end{minipage} \\
\midrule\noalign{}
\endhead
\bottomrule\noalign{}
\endlastfoot
101 & Database & Only a joke \\
102 & Discrete Math & Excellent course \\
\end{longtable}

Table 3 Comment

\begin{enumerate}
\def\labelenumi{\alph{enumi})}
\item
  SELECT * FROM Enroll, Comment WHERE Enroll.course = Comment.course
\item
  SELECT * FROM Student NATURAL JOIN Enroll NATURAL JOIN Comment
\item
  SELECT * FROM Enroll NATURAL RIGHT OUTER JOIN Comment
\item
  SELECT sID, course FROM Enroll JOIN Comment
\end{enumerate}

\end{document}
