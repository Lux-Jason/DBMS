% Options for packages loaded elsewhere
\PassOptionsToPackage{unicode}{hyperref}
\PassOptionsToPackage{hyphens}{url}
%
\documentclass[
]{article}
\usepackage{amsmath,amssymb}
\usepackage{iftex}
\ifPDFTeX
  \usepackage[T1]{fontenc}
  \usepackage[utf8]{inputenc}
  \usepackage{textcomp} % provide euro and other symbols
\else % if luatex or xetex
  \usepackage{unicode-math} % this also loads fontspec
  \defaultfontfeatures{Scale=MatchLowercase}
  \defaultfontfeatures[\rmfamily]{Ligatures=TeX,Scale=1}
\fi
\usepackage{lmodern}
\ifPDFTeX\else
  % xetex/luatex font selection
\fi
% Use upquote if available, for straight quotes in verbatim environments
\IfFileExists{upquote.sty}{\usepackage{upquote}}{}
\IfFileExists{microtype.sty}{% use microtype if available
  \usepackage[]{microtype}
  \UseMicrotypeSet[protrusion]{basicmath} % disable protrusion for tt fonts
}{}
\makeatletter
\@ifundefined{KOMAClassName}{% if non-KOMA class
  \IfFileExists{parskip.sty}{%
    \usepackage{parskip}
  }{% else
    \setlength{\parindent}{0pt}
    \setlength{\parskip}{6pt plus 2pt minus 1pt}}
}{% if KOMA class
  \KOMAoptions{parskip=half}}
\makeatother
\usepackage{xcolor}
\usepackage{longtable,booktabs,array}
\usepackage{calc} % for calculating minipage widths
% Correct order of tables after \paragraph or \subparagraph
\usepackage{etoolbox}
\makeatletter
\patchcmd\longtable{\par}{\if@noskipsec\mbox{}\fi\par}{}{}
\makeatother
% Allow footnotes in longtable head/foot
\IfFileExists{footnotehyper.sty}{\usepackage{footnotehyper}}{\usepackage{footnote}}
\makesavenoteenv{longtable}
\setlength{\emergencystretch}{3em} % prevent overfull lines
\providecommand{\tightlist}{%
  \setlength{\itemsep}{0pt}\setlength{\parskip}{0pt}}
\setcounter{secnumdepth}{-\maxdimen} % remove section numbering
\ifLuaTeX
  \usepackage{selnolig}  % disable illegal ligatures
\fi
\usepackage{bookmark}
\IfFileExists{xurl.sty}{\usepackage{xurl}}{} % add URL line breaks if available
\urlstyle{same}
\hypersetup{
  hidelinks,
  pdfcreator={LaTeX via pandoc}}

\author{}
\date{}

\begin{document}

COMP3013 2024 Fall

Assignment 3

For the submission, please pack all files and convert them into
\textbf{A} \textbf{SINGLE PDF FILE}. Rename the PDF file as
COMP3013\_24F\_A3\_XXXX, where XXXX is your student ID.

The schema of a database for public transportation companies is as
follows. Keys are underlined.

\begin{itemize}
\item
  \(company = (\underline{cID},cname,address,phone)\)
\item
  \(route = (\underline{rID},departure,arrival,cID)\)
\end{itemize}

// \(cID\) is a foreign key to \(company.cID\).

\begin{itemize}
\item
  \(vehicle = (\underline{plateNum},model,capacity,manufacturer,cID)\)
\end{itemize}

// \(cID\) is a foreign key to \(company.cID\).

\begin{itemize}
\item
  \(driver = (\underline{dID},name,\ gender,age,cID)\)
\end{itemize}

// \(cID\) is a foreign key to \(company.cID\).

\begin{itemize}
\item
  \(serve = (\underline{rID,plateNum})\)
\end{itemize}

// \(rID\) is a foreign key to \(route.rID\).

// \(plateNum\) is a foreign key to \(vehicle.plateNum\).

\begin{itemize}
\item
  \(drive = (\underline{plateNum},dID)\)
\end{itemize}

// \(dID\) is a foreign key to \(diver.dID\).

// \(plateNum\) is a foreign key to \(vehicle.plateNum\).

Q1. Write a query for each following question. (10pt for each)

\begin{enumerate}
\def\labelenumi{\alph{enumi})}
\item
  Find the ID of routes which are not served by any vehicle. You must
  use subqueries.


\item
  Find the name of drivers who have served the route 69 (rID). You must
  use subqueries.


\item
  Find the name of drivers who have driven all vehicles.


\item
  Find the plate number of vehicles which have served all route operated
  by ``Xinhe'' (company name).


\item
  Implement constraints to guarantee the gender of a driver is either
  ``Male'' or ``Female'' and the age is from 20 to 60.

  
\end{enumerate}

Q2. Given an instance of a relational schema \(R = \{ A,B,C\}\) and a
list of functional dependencies.

\begin{longtable}[]{@{}
  >{\raggedright\arraybackslash}p{(\columnwidth - 4\tabcolsep) * \real{0.3314}}
  >{\raggedright\arraybackslash}p{(\columnwidth - 4\tabcolsep) * \real{0.3378}}
  >{\raggedright\arraybackslash}p{(\columnwidth - 4\tabcolsep) * \real{0.3307}}@{}}
\toprule\noalign{}
\begin{minipage}[b]{\linewidth}\raggedright
\[A\]
\end{minipage} & \begin{minipage}[b]{\linewidth}\raggedright
\[B\]
\end{minipage} & \begin{minipage}[b]{\linewidth}\raggedright
\[C\]
\end{minipage} \\
\midrule\noalign{}
\endhead
\bottomrule\noalign{}
\endlastfoot
1 & 2 & 2 \\
1 & 3 & 2 \\
1 & 4 & 2 \\
2 & 5 & 2 \\
\end{longtable}

Decide whether the functional dependency is satisfied by the instance.
(10 pt)

a) \(A \rightarrow B\) b) \(A \rightarrow C\) c) \(B \rightarrow A\)

d) \(B \rightarrow C\) e) \(C \rightarrow A\) f) \(C \rightarrow B\)

g) \(AB \rightarrow C\) h) \(AC \rightarrow B\) i) \(BC \rightarrow A\)

Q3. Given a relational schema and a set of functional dependencies

\begin{itemize}
\item
  \(R = \{ A,B,C,D,E\}\)
\item
  \(F = \{ AB \rightarrow CD,\ BC \rightarrow DE,CD \rightarrow E,DE \rightarrow A\}\)
\end{itemize}

\begin{enumerate}
\def\labelenumi{\alph{enumi})}
\item
  Find all candidate keys of \(R\). (6 pt)
\item
  Decompose \(R\) into BCNF. Show the steps. (15 pt)
\item
  Does the BCNF decomposition in part b) preserve all functional
  dependencies? (4 pt)
\item
  Decompose \(R\) into 3NF. Show the steps. (15 pt)
\end{enumerate}

\end{document}
