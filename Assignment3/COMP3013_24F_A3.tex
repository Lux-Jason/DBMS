\documentclass[12pt]{article}
\usepackage{amsmath, amssymb, longtable}
\usepackage{booktabs}
\usepackage{array}
\usepackage{enumitem}

\title{COMP3013 2024 Fall\\Assignment 3}
\author{Qiao Yichang}
\date{2024/12/13}

\begin{document}
	\maketitle
	
	\section*{Schema of the Database}
	The schema of a database for public transportation companies is as follows. Keys are underlined.
	
	\begin{itemize}
		\item \(company = (\underline{cID}, cname, address, phone)\)
		\item \(route = (\underline{rID}, departure, arrival, cID)\)
		\item \(vehicle = (\underline{plateNum}, model, capacity, manufacturer, cID)\)
		\item \(driver = (\underline{dID}, name, gender, age, cID)\)
		\item \(serve = (\underline{rID, plateNum})\)
		\item \(drive = (\underline{plateNum}, dID)\)
	\end{itemize}
	
	\noindent \textit{Notes:}
	\begin{itemize}
		\item \(cID\) is a foreign key to \(company.cID\).
		\item \(rID\) is a foreign key to \(route.rID\).
		\item \(plateNum\) is a foreign key to \(vehicle.plateNum\).
		\item \(dID\) is a foreign key to \(driver.dID\).
	\end{itemize}
	
	\section*{Q1.}
	\begin{enumerate}
		\item[(a)] Find the ID of routes which are not served by any vehicle. You must use subqueries.

		\begin{verbatim}
			SELECT rID 
			FROM route 
			WHERE rID NOT IN (SELECT rID FROM serve);
		\end{verbatim}
		
		\item[(b)] Find the name of drivers who have served the route 69 (rID). You must use subqueries.

		\begin{verbatim}
			SELECT name 
			FROM driver 
			WHERE dID IN (
			SELECT dID 
			FROM drive 
			WHERE plateNum IN (
			SELECT plateNum 
			FROM serve 
			WHERE rID = 69
			)
			);
		\end{verbatim}
		
		\item[(c)] Find the name of drivers who have driven all vehicles.

		\begin{verbatim}
			SELECT name 
			FROM driver d 
			WHERE NOT EXISTS (
			SELECT plateNum 
			FROM vehicle 
			WHERE plateNum NOT IN (
			SELECT plateNum 
			FROM drive 
			WHERE drive.dID = d.dID
			)
			);
		\end{verbatim}
		
		\item[(d)] Find the plate number of vehicles which have served all routes operated by ``Xinhe'' (company name).

		\begin{verbatim}
			SELECT plateNum 
			FROM vehicle v 
			WHERE NOT EXISTS (
			SELECT rID 
			FROM route 
			WHERE cID = (SELECT cID FROM company WHERE cname = 'Xinhe')
			AND rID NOT IN (
			SELECT rID 
			FROM serve 
			WHERE serve.plateNum = v.plateNum
			)
			);
		\end{verbatim}
		
		\item[(e)] Implement constraints to guarantee the gender of a driver is either ``Male'' or ``Female'' and the age is from 20 to 60.

		\begin{verbatim}
			ALTER TABLE driver
			ADD CONSTRAINT chk_gender CHECK (gender IN ('Male', 'Female'));
			
			ALTER TABLE driver
			ADD CONSTRAINT chk_age CHECK (age BETWEEN 20 AND 60);
		\end{verbatim}
		
	\end{enumerate}
	
	\section*{Q2.}
Given an instance of a relational schema \( R = \{ A, B, C \} \) and the following table:

\begin{table}[h]
	\centering
	\begin{tabular}{ccc}
		\toprule
		A & B & C \\
		\midrule
		1 & 2 & 2 \\
		1 & 3 & 2 \\
		1 & 4 & 2 \\
		2 & 5 & 2 \\
		\bottomrule
	\end{tabular}
\end{table}

Decide whether each of the following functional dependencies is satisfied by the instance:


a) \( A \rightarrow B \): Does not hold. For \( A = 1 \), \( B \) has different values (2, 3, 4).

b) \( A \rightarrow C \): Holds. For each \( A \), \( C \) is consistently 2.

c) \( B \rightarrow A \): Holds. Each \( B \) value corresponds to a unique \( A \) value.

d) \( B \rightarrow C \): Holds. For all \( B \) values, \( C \) is always 2.

e) \( C \rightarrow A \): Does not hold. Same \( C \) (2) corresponds to different \( A \) values (1 and 2).

f) \( C \rightarrow B \): Does not hold. Same \( C \) (2) corresponds to different \( B \) values (2, 3, 4, 5).

g) \( AB \rightarrow C \): Holds. Each \( AB \) pair corresponds to \( C = 2 \).

h) \( AC \rightarrow B \): Does not hold. Same \( AC \) (1,2) corresponds to different \( B \) values (2, 3, 4).

i) \( BC \rightarrow A \): Holds. Each \( BC \) pair corresponds to a unique \( A \) value.

	\section*{Q3.}
\begin{itemize}
	\item \( R = \{ A, B, C, D, E \} \)
	\item \( F = \{ AB \rightarrow CD, BC \rightarrow DE, CD \rightarrow E, DE \rightarrow A \} \)
\end{itemize}

\begin{enumerate}
	\def\labelenumi{\alph{enumi})}
	
	\item \textbf{Find all candidate keys of \( R \). (6 pt)}
	
Analyze:

	To find the candidate keys, we compute the closure of various sets of attributes.
	

	\[
	AB^+ = \{ A, B, C, D, E \}
	\]
	The closure of \( AB \) contains all the attributes in \( R \), so \( AB \) is a candidate key.
	

	Other smaller combinations of attributes do not contain all attributes in their closures, so \( AB \) is the only candidate key.
	
Answer:

	The candidate key is \( AB \).
	
	\item \textbf{Decompose \( R \) into BCNF. Show the steps. (15 pt)}
	

	A relation is in BCNF if for every functional dependency \( X \rightarrow Y \), \( X \) is a superkey. Let's check each dependency:
	
	\begin{itemize}
		\item \( AB \rightarrow CD \): \( AB \) is a superkey, so this satisfies BCNF.
		\item \( BC \rightarrow DE \): \( BC \) is not a superkey, so this violates BCNF.
	\end{itemize}
	
	Since \( BC \rightarrow DE \) violates BCNF, we need to decompose \( R \).
	

	\begin{itemize}
		\item Create a relation for \( BC \rightarrow DE \): \( R_1(B, C, D, E) \).
		\item Create a relation for the remaining attributes: \( R_2(A, B) \).
	\end{itemize}
	

	\begin{itemize}
		\item \( R_1(B, C, D, E) \) satisfies BCNF because \( BC \) is a superkey.
		\item \( R_2(A, B) \) satisfies BCNF because \( AB \) is a superkey.
	\end{itemize}
	
Answer:

	 The BCNF decomposition is:
	\[
	R_1(B, C, D, E), \quad R_2(A, B)
	\]
	
	\item \textbf{Does the BCNF decomposition in part b) preserve all functional dependencies? (4 pt)}
	
Answer:

	The BCNF decomposition does not preserve all functional dependencies. Specifically, the following dependencies are not preserved:
	\begin{itemize}
		\item \( CD \rightarrow E \)
		\item \( DE \rightarrow A \)
	\end{itemize}
	
	These dependencies cannot be directly enforced in the decomposed relations.
	
	\item \textbf{Decompose \( R \) into 3NF. Show the steps. (15 pt)}
	
	A relation is in 3NF if for every functional dependency \( X \rightarrow Y \), at least one of the following holds:
	\begin{enumerate}
		\item \( X \) is a superkey, or
		\item Every attribute in \( Y \) is a prime attribute (part of a candidate key).
	\end{enumerate}
	
	Checking each dependency:
	\begin{itemize}
		\item \( AB \rightarrow CD \): Satisfied since \( AB \) is a superkey.
		\item \( BC \rightarrow DE \): Violates 3NF because \( BC \) is not a superkey, and \( D \) and \( E \) are non-prime.
		\item \( CD \rightarrow E \): Violates 3NF because \( CD \) is not a superkey, and \( E \) is non-prime.
		\item \( DE \rightarrow A \): Violates 3NF because \( DE \) is not a superkey, and \( A \) is non-prime.
	\end{itemize}

	\begin{itemize}
		\item Create a relation \( R_1(B, C, D, E) \) to handle \( BC \rightarrow DE \).
		\item Create a relation \( R_2(C, D, E) \) to handle \( CD \rightarrow E \).
		\item Create a relation \( R_3(D, E, A) \) to handle \( DE \rightarrow A \).
	\end{itemize}
	
	\begin{itemize}
		\item \( R_1(B, C, D, E) \) satisfies 3NF because \( BC \) is a superkey.
		\item \( R_2(C, D, E) \) satisfies 3NF because \( CD \) is a superkey.
		\item \( R_3(D, E, A) \) satisfies 3NF because \( DE \) is a superkey.
	\end{itemize}
	
Answer:

	\[
	R_1(B, C, D, E), \quad R_2(C, D, E), \quad R_3(D, E, A)
	\]
	
\end{enumerate}

	
\end{document}
